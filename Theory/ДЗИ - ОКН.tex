\documentclass{article}
\usepackage{amsmath}
\usepackage{amssymb}
\usepackage{amsthm}
\usepackage{mathabx}
\usepackage{parskip}
\usepackage{mathtools}
\usepackage[utf8]{inputenc}
\usepackage[bulgarian]{babel}

\begin{document}

\title{Теми за Държавен Изпит}
\author{}
\maketitle

\section*{Дискретни структури}

\subsection*{Множества. Декартово произведение. Релации. Функции.}

\textbf{\underline{Аксиома} (за обема)} \newline $\forall x (x \in A \iff x \in B) \implies A = B$ \newline\newline
\textbf{\underline{Аксиома} (за отделянето)} \newline $A' = \{x | x \in A, \pi(x)\}$ е множество \newline\newline
\textbf{\underline{Аксиома} (за степенното множество)} \newline Съвкупността от всички подмножества на множеството
X е множество \newline\newline
\textbf{\underline{Аксиома} (за индукцията)} \newline $X = <X_0, \mathcal{F}>$ е множество, където $X_0$ е непразно
множество, а $\mathcal{F}$ е множество от операции \newline\newline
\textbf{\underline{Теорема} (математическа индукция)} \newline
За всеки елемент $x$ на индуктивно дефинираното мн-во $M$ е в сила $\pi(x)$
\begin{proof}
    $ $\newline
    \begin{itemize}
        \item База: За всеки базов елемент $x_0$ от $M_0$ проверяваме верността на $\pi(x_0)$
        \item Индукционно предположение: Допускаме, че $\pi(x)$ е в сила за всеки елемент $x$, включен до определен
        момент в множеството $M$
        \item Индукционна стъпка: Показваме, че при направеното предположение, за всеки построен с помощта на
        операциите от $\mathcal{F}$ елемент $y$ на $M$, също е в сила $\pi(y)$
        \item Заключение: $\pi$ е в сила за всеки елемент на множеството $M$
    \end{itemize}
\end{proof}
\textbf{Основни операции} върху множества и техните свойства:
$$A \cup B = \{x | x \in A \text{ или } x \in B\}$$
$$A \cap B = \{x | x \in A \text{ и } x \in B\}$$
$$A \backslash B = \{x | x \in A \text{ и } x \notin B\}$$
$$A \triangle B = \{x | (x \in A \text{ и } x \notin B) \text{ или } (x \notin A \text{ и } x \in B)\}$$
1. Идемпотентност - $A \cup A = A, A \cap A = A$ \newline
2. Комутативност - $A \cup B = B \cup A, A \cap B = B \cap A, A \triangle B = B \cup A$ \newline
3. Асоциативност - $(A \cup B) \cup C = A \cup (B \cup C), (A \cap B) \cap C = A \cap (B \cap C),
(A \triangle B) \triangle C = A \triangle (B \triangle C)$ \newline
4. Дистрибутивност - $(A \cup B) \cap C = (A \cap C) \cup (B \cap C), (A \cap B) \cup C = (A \cup C) \cap
(B \cup C)$ \newline
5. Свойства на празното и универсалното множество - $A \cup \varnothing = A, A \cap \varnothing = \varnothing,
A \cup U = U, A \cap U = A$ \newline
6. Свойства на допълнението - $A \cup \overline{A} = U, A \cap \overline{A} = \varnothing, \overline{\overline{A}} = A$ \newline
7. Закони на Де Морган - $\overline{A \cup B} = \overline{A} \cap \overline{B}, \overline{A \cap B} = \overline{A}
\cup \overline{B}$ \newline\newline
\textbf{\underline{Дефиниция}} Нека $a$ и $b$ са произволни елементи. Означаваме с $(a, b)$ множеството
$\{a, \{a, b\}\}$ и го наричаме \textbf{наредена двойка} от елементите $a$ и $b$ \newline\newline
\textbf{\underline{Дефиниция}} Нека $A$ и $B$ са множества. Множеството 
$A \bigtimes B = \{(a, b) | a \in A, b \in B\}$ наричаме \textbf{Декартово произведение} на множествата $A$ и $B$ \newline\newline
\textbf{\underline{Дефиниция}}
Нека $n \in \mathbb{N}$ и $A_1,...,A_n$ е фамилия от множества. Всяко $R \subseteq A_1 \bigtimes ... \bigtimes A_n$ наричаме
\textbf{$n$-местна релация} над декартовото произведение $A_1 \bigtimes ... \bigtimes A_n$ или просто \textbf{$n$-местна релация}.
Множествата $A_1, ..., A_n$ наричаме домейни. В случай, че $n = 2$, релацията наричаме двуместна, или просто \textbf{релация}. \newline\newline
\textbf{\underline{Дефиниция}}
Нека $R \subseteq A \bigtimes A$. Казваме, че релацията $R$ е:
\begin{itemize}
    \item \textbf{Рефлексивна} - ако $\forall a \in A, (a, a) \in R$
    \item \textbf{Симетрична} - ако $\forall a, b \in A, a \neq b, (a, b) \in R \implies (b, a) \in R$
    \item \textbf{Транзитивна} - ако $\forall a, b, c \in A, (a, b) \in R, (b, c) \in R \implies (a, c) \in R$
    \item \textbf{Антирефлексивна} - ако $\forall a \in A, (a, a) \notin R$
    \item \textbf{Антисиметрична} - ако $\forall a, b \in A, a \neq b, (a, b) \in R \implies (b, a) \notin R$
    \item \textbf{Силно антисиметрична} - ако $\forall a, b \in A, a \neq b$, точно едно от $(a, b) \in R$ или $(b, a) \in R$ е в сила
\end{itemize}
\textbf{\underline{Дефиниция}}
Релацията $R \subseteq A \bigtimes A$ наричаме \textbf{еквивалентност}, ако е рефлексивна, симетрична и транзитивна. \newline\newline
\textbf{\underline{Дефиниция}}
Нека $R \subseteq A \bigtimes A$ е релация на еквивалентност, $a \in A$ и $[a] = \{b | b \in A, aRb\}$.
Нека фамилията $\mathcal{R} = \{[a_i] | i \in I\}$ се състои от всички различни множества $[a_i]$, където
$I$ е подходящо индексно множество. Тогава $\mathcal{R}$ е разбиране на $A$. Множествата $[a_i]$ наричаме
\textbf{класове на еквивалентност}. \newline\newline
\textbf{\underline{Дефиниция}}
Релацията $R \subseteq A \bigtimes A$ наричаме \textbf{частична наредба}, ако е рефлексивна, антисиметрична и транзитивна. \newline\newline
\textbf{\underline{Дефиниция}}
Частичната наредба $R \subseteq A \bigtimes A$ наричаме \textbf{пълна(линейна)}, ако силно антисиметрична. \newline\newline
\textbf{\underline{Дефиниция}}
Нека $R \subseteq A \bigtimes A$. Елементът $a \in A$ наричаме \textbf{минимален} в $R$, ако $\nexists b \in A, b \neq a$ и
такъв, че $bRa$. Елементът $a \in A$ наричаме \textbf{максимален} в $R$, ако $\nexists b \in A, b \neq a$ и такъв, че $aRb$. \newline\newline
\textbf{\underline{Алгоритъм} (топологично сортиране)} \newline\newline
\underline{Дадено}: Множество $A = \{a_1,a_2,...,a_n\}$ и частична наредба $R \subseteq A \bigtimes A$. \newline
\underline{Резултат}: Наредена $n$-торка $(a_{i_1},a_{i_2},...,a_{i_n})$ от различните елементи на $A$, задаваща пълна наредба
$R' \subseteq A \bigtimes A, R \subseteq R': \forall m \in I_n, \forall k \in I_n, m \le k \rightarrow (a_{i_m}, a_{i_k}) \in R'$. \newline
\underline{Процедура}: \begin{enumerate}
    \item $j = 1, T = R, M = A$
    \item Намираме $a_{i_j}$ - минимален в $M$ за релацията $T$
    \item $T = T \backslash \{(a_{i_j}, a_l) | (a_{i_j}, a_l) \in T\}; M = M \backslash \{a_{i_j}\}; j = j + 1$
    \item Ако $M \neq \varnothing$ преминаваме към $2.$, в противен случай алгоритъмът спира.
\end{enumerate}
\textbf{\underline{Дефиниция}}
Релацията $f \subseteq X \bigtimes Y$ наричаме \textbf{частична функция}, ако $\forall a \in X \exists$ не повече от 
едно $b \in Y : (a, b) \in f$ \newline\newline
\textbf{\underline{Дефиниция}}
Релацията $f \subseteq X \bigtimes Y$ наричаме \textbf{тотална функция}, ако $\forall a \in X \exists$ точно
едно $b \in Y : (a, b) \in f$ \newline\newline
\textbf{\underline{Дефиниция}}
Ако $f(x_1) \neq f(x_2), \forall x_1 \neq x_2 \in X$, тогава функцията $f$ наричаме \textbf{инекция}. \newline\newline
\textbf{\underline{Дефиниция}}
Ако $\forall b \in Y, \exists a \in X, f(a) = b$, тогава функцията $f$ наричаме \textbf{сюрекция} на $X$ върху $Y$. \newline\newline
\textbf{\underline{Дефиниция}}
Ако тоталната функция $f : X \rightarrow Y$ е инекция и сюрекция едновременно, казваме, че $f$ е \textbf{биекция}. \newline\newline
\textbf{\underline{Дефиниция}}
Множеството $A$ е \textbf{крайно}, ако $A = \varnothing$ или $\exists n \in N, n \ge 1$ и биекция $f : A \rightarrow I_n$
$|A| = 0$, ако $A = \varnothing$ и $|A| = n$ в противен случай, наричаме брой на елементите на $A$ (\textbf{кардиналност}). \newline\newline
\textbf{\underline{Дефиниция}}
Казваме, че множеството $A$ е \textbf{изброимо безкрайно}, ако $\exists \text { биекция } f : A \rightarrow N$.
Казваме, че множеството $A$ е \textbf{изброимо}, ако е крайно или изброимо безкрайно. \newline\newline
\textbf{\underline{Принцип на Дирихле}}
Нека $X$ и $Y$ са крайни множества и $|X| > |Y|$. Тогава за всяка тотална функция $f : X \rightarrow Y$
съществуват $x_1 \neq x_2 \in X$ такива, че $f(x_1) = f(x_2)$

\subsection*{Основни комбинаторни принципи и конфигурации. Рекурентни уравнения.}

\textbf{\underline{Теорема} (принцип на Дирихле)} \newline
Нека $X$ е множество с $n$ елемента (предмета), а $Y$ е множество с $m$ елемента (чекмеджета) и $n > m$. Както и да поставим
всички предмети в чекмеджетата, поне в едно чекмедже ще има поне два предмета. \newline\newline
\textbf{\underline{Теорема} (принцип на биекцията)} \newline
Нека $X$ и $Y$ са крайни множества, $|X| = n, |Y| = m$. Съществува биекция $f : X \rightarrow Y \iff n = m$ \newline\newline
\textbf{\underline{Теорема} (принцип на събирането)} \newline
Нека $A$ е крайно множество, а $\mathcal{R} = \{S_1, ..., S_n\}$ е разбиване на $A$. Тогава $|A| = \sum_{i = 1}^{k}|S_i|$ \newline\newline
\textbf{\underline{Теорема} (принцип на изваждането)} \newline
Нека $A$ е крайно множество, $A', A'' \subseteq A, A' = A \ A''$. Тогава $|A'| = |A| - |A''|$ \newline\newline
\textbf{\underline{Теорема} (принцип на умножението)} \newline
Нека $X$ и $Y$ са крайни множества, $|X| = n, |Y| = m$. Тогава $|X \bigtimes Y| = |X| . |Y| = nm$ \newline\newline
\textbf{\underline{Теорема} (принцип на делението)} \newline
Нека $A$ е крайно множество и $B = A \bigtimes C$, където $C$ също е крайно, $|C| \neq 0$. 
Тогава $|A| = |B| / |C|$ \newline\newline
\textbf{\underline{Теорема} (принцип за включване и изключване)} \newline
Нека $A$ е крайно множество и $A_1, ..., A_n \subseteq A$. Тогава:
$$|A_1| = |A| - \sum|A_i| + \sum|A_i \cap A_j| - \sum|A_i \cap A_j \cap A_k| + ... + (-1)^n|A_1 \cap A_2 \cap ... \cap A_n|$$
\textbf{\underline{Комбинаторни конфигурации с наредба и повторение}} \newline\newline
Нека $0 < m \in \mathbb{Z}$ и $\mathcal{K}_{\text{н,п}}(n, m)$ е множеството, в което всеки елемент е наредена $m$-торка с
елементи от образуващото множество $A$, като един елемент на $A$ може да участва произволен брой пъти в $m$-торката.
$|\mathcal{K}_{\text{н,п}}(n, m)| = |A^m| = n^m$ \newline\newline
\textbf{\underline{Комбинаторни конфигурации с наредба и без повторение}} \newline\newline
Нека $1 \le m \le n \in \mathbb{Z}$ и $\mathcal{K}_{\text{н}}(n, m)$ е множеството от наредените $m$-торки с елементи от образуващото
множество $A$, в които всеки елемент участва не повече от веднъж. Нека $m \ge 2$. Тогава на първо място в $m$-торката може да се
постави кой да е от елементите на $A$, които са $n$ на брой, а на второ място всеки от останалите $n - 1$. Съгласно принципа за
умножението, за първите позиции в $m$-торката получаваме $n(n - 1)$ възможности. За всяка от тях можем да изберем трети елемент
по $(n - 2)$ възможни начина и съгласно Принципа за умножението, за първите 3 позиции имаме $n(n - 1)(n - 2)$ възможности.
Разсъждавайки индуктивно, получаваме за броя на конфигурациите с наредба и без повтаряне $\mathcal{K}_{\text{н}}(n, m) = n(n - 1)
...(n - m + 1)$. В случай, че $m = 1$ имаме точно $n$ възможности и тъй като $n = \mathcal{K}_{\text{н}}(n, 1)$, формулата е в
сила и в този случай. \newline
Комбинаторните конфигурации от $m$ елемента, построени от множество с $n$ елемента, $1 \le m \le n$, с наредба и без повтаряне
се наричат още \textbf{вариации} на $n$ елемента от $m$-ти клас и се означават с $V_n^m$\newline
При $m = n$, комбинаторната конфигурация се нарича \textbf{пермутация} и $|P_n| = \mathcal{K}_{\text{н}}(n, n) = n(n - 1)...2.1
= n!$ \newline\newline
\textbf{\underline{Комбинаторни конфигурации без наредба и без повторение}} \newline\newline
Нека $0 > n, 0 \le m \le n \in \mathbb{Z}$ и $\mathcal{K}(n, m)$ е множеството от наредените $m$-торки, без повтаряне, от елементи
на образуващото множество $A$ с $n$ елемента. Всъщност отсъствието на наредба и повтаряне означава, че разглежданите
конфигурации са $m$-елементните подмножества на $A$. Наричаме ги \textbf{комбинации} на $n$ елемента от $m$-ти клас и ги
означаваме с $C_n^m$. За да намерим броя на комбинациите на $n$ елемента от $m$-ти клас, ще използваме Принципа на делението.
Ако вместо ненаредените $m$-торки без повтаряне разгледаме наредените, множеството $\mathcal{K}(n, m)$ ще се разшири до
$\mathcal{K'}(n, m) = \mathcal{K_{\text{н}}}(n, m)$, при това всяка ненаредена $m$-орка ще се среща в новото множество по толкова
начина, по колкото можем да подредим всичките й елементи, т.е. по $m!$ начина. Следователно $|\mathcal{K}(n, m)| = \frac{\mathcal{K_{\text{н}}}
(n, m)}{m!} = \frac{n(n - 1) ... (n - m + 1)}{m!} = \frac{n!}{m!(n - m)!}$. Така получения израз означаваме с $\binom{n}{m}$ и
наричаме биномен коефициент. \newline\newline
\textbf{\underline{Теорема} (Нютон)}
$$\forall n \in \mathbb{N}, \forall x, y \in \mathbb{R} : (x + y)^n = \sum_{m = 0}^{n}\binom{n}{m}x^my^{n-m}$$
\begin{proof}
При повдигане бинома $x + y$ на $n$-та степен всеки едночлен от вида $x^my^{n - m}$ ще се получи толкова пъти, по колкото начина
можем да изберем $m$ множители измежду $n$-те, които взимаме $x$ (от останалите $n - m$ избираме $y$), а това е точно броят на
ненаредените конфигурации без повторение $\binom{n}{m}$
\end{proof}
В комбинаториката едно тъждество може да бъде доказано както по формален път, така и по комбинаторен път, чрез преброяване на
елементите на подходящо избрана конфигурация по два различни начина. Тази техника е известна като \textbf{принцип на двукратното
броене}. \newline\newline
\textbf{\underline{Теорема}} \newline
Нека $n, m \in \mathbb{N}, n > m$. Тогава $\binom{n}{m} = \binom{n - 1}{m} + \binom{n - 1}{m - 1}$ и $\binom{n}{m} = \binom{n}{n - m}$ \newline\newline
\textbf{\underline{Теорема}} \newline
Нека $n, m \in \mathbb{N}$. Тогава $\binom{n + m}{m} = \sum_{k = 0}^{m}\binom{n + k - 1}{k}$ \newline\newline
\textbf{\underline{Комбинаторни конфигурации без наредба и с повторение}} \newline\newline

\subsection*{Графи. Дървета. Обхождания на графи.}

\textbf{\underline{Дефиниция}}
Нека $V = \{v_1, ..., v_n\}$ е крайно множество, елементите на което наричаме върхове, а $E = \{e_1, ..., e_m\}$
е крайно множество, елементите на което наричаме ребра. Функцията $f_G : E \rightarrow V \bigtimes V$ съпоставяща
на всяко ребро наредена двойка от върхове, наричаме \textbf{краен ориентиран мултиграф}. \newline\newline
\textbf{\underline{Дефиниция}}
Нека $G(V, E, f_G)$ е краен ориентиран мултиграф и функцията $f_G$ е еднозначна, т.е. $f(e_i) \neq f(e_j), i \neq j$.
Тогава $G(V, E, f_G)$ наричаме \textbf{краен ориентиран граф}. \newline\newline
\textbf{\underline{Дефиниция}}
Нека $G(V, E)$ е краен ориентиран граф, такъв че релацията $E \subseteq V \bigtimes V$ е антирефлексивна и 
симетрична. Тогава $G(V, E)$ наричаме \textbf{краен неориентиран граф} или просто \textbf{граф}. \newline\newline
\textbf{\underline{Дефиниция}}
Нека $G(V, E)$ е мултиграф. Последователността от върхове $v_{i_0}...v_{i_l}$ наричаме \textbf{път} в $G$ от $v_{i_0}$ до
$v_{i_l}$, ако $(v_{i_j}, v_{i_{j + 1}}) \in E$ и $v_{i_{j - 1}} \neq v_{i_{j + 1}}, \forall j$. Числото l наричаме
\textbf{дължина на пътя}. В случая, когато $v_{i_0} = v_{i_l}$, пътя наричаме \textbf{цикъл}. \newline\newline
\textbf{\underline{Дефиниция}}
За всеки краен неориентиран граф $G(V, E)$ дефинираме релацията $\mathcal{P}_G \subseteq V \bigtimes V$ така: $(v_i, v_j) \in
\mathcal{P}_G \iff \exists$ път в $G$ от $v_i$ до $v_j$. Релацията $\mathcal{P}_G$ е релация на еквивалентност. \newline\newline
\textbf{\underline{Дефиниция}}
Върховете от всеки клас на еквивалентност $V' \subseteq V$ на релацията $\mathcal{P}_G$ индуцират в $G(V, E)$ по един подграф,
който наричаме \textbf{свързана компонента}. Графът $G(V, E)$ се нарича \textbf{свързан}, ако $\forall v_i, v_j \in V$, $\exists$
път в $G$ от $v_i$ до $v_j$. \newline\newline
\textbf{\underline{Дефиниция}}
Свързан граф без цикли наричаме \textbf{дърво}, а несвързан граф без цикли - \textbf{гора}. \newline\newline
\textbf{\underline{Дефиниция}}
Графът $D(\{r\}, \{\})$ с един връх $r$ и без ребра наричаме дърво с корен $r$ (\textbf{кореново дърво}). Единственият
връх $r$ е единствен лист на това кореново дърво. \newline\newline
\textbf{\underline{Теорема}}
Всяко дърво с корен е дърво.
\begin{proof}
Прилагаме индукция по дефиницията на дърво с корен.
\begin{itemize}
    \item Тривиалното дърво $D(\{r\}, \{\})$ е очевидно, че е свързан граф $\implies$ тривиалното кореново дърво е дърво.
    \item Допускаме, че дървото $D(V, E)$, с корен $r$ е свързан граф без цикли.
    \item Ще докажем, че дървото $D'(V', E') = D'(V \cup \{w\}, E \cup \{(u, w)\})$ с корен $r$ е свързан граф. \newline
    За произволни два върха $v_i$ и $v_j$ от $V \exists$ път от $v_i$ до $v_j$ в $D$, защото според ИП $D$ е свързан граф
    $\implies \exists$ такъв път и в $D'$. Път от произволен връх $v_i \in V$ до $w$ в $D'$ получаваме, като към съществуващия
    в $D$ път от $v_i$ до $u$ добавим реброто $(u, w)$, с което $w$ е $u$, защото $w \notin V \implies \exists$ път в $D'$ между
    всеки два върха и $D'$ е свързан граф. \newline Съгласно ИП в $D$ няма цикли, а за да има цикъл в $D'$, той трябва да съдържа
    реброто $(u, w)$. Това не е възможно, тъй като $w$ е край само на реброто $(u, w), |w| = 1$ и съгласно дефиницията не може
    да участва в цикъл.
\end{itemize}
\end{proof}
\textbf{\underline{Теорема}}
Нека $D(V, E)$ е дърво с корен $r$. Тогава $|V| = |E| + 1$.
\begin{proof}
    $\\$
    \begin{itemize}
        \item За тривиалното кореново дърво $D(V = {r}, E = \{\})$ имаме $|V| = 1, |E| = 0 \implies |V| = |E| + 1$
        \item Нека $D(V, E)$ е кореново дърво и $|V| = |E| + 1$
        \item Присъединяваме $w$ към върха $u \in V$ и получаваме кореновото дърво $D'(V', E'), V' = V \cup {w}, E' = E \cup
        {(u, w)}$. Сега $|V'| = |V| + 1, |E'| = |E| + 1 \implies |V'| - |E'| = |V| - |E| = 1$ и $|V'| = |E'| + 1$
    \end{itemize}
\end{proof}
\textbf{\underline{Дефиниция}}
Нека $G(V, E)$ е граф, а $D(V, E'), E' \subseteq E$ е дърво. Тогава $D$ наричаме \textbf{покриващо дърво} на $G$. \newline\newline
\textbf{\underline{Алгоритъм} (обхождане в ширина)} \newline\newline
\underline{Дадено}: Свързан граф $G(V, E)$ \newline
\underline{Резултат}: Покриващо дърво $D(V, E')$ на $G$. \newline
\underline{Процедура}:
\begin{enumerate}
    \item Коренът $r$ на покриващото дърво ще изберем за начален връх на обхождането. Затова $L_0 = \{r\}$. Образуваме дървото
    $D_0(V_0, E_0), V_0 = L_0, E_0 = \varnothing$ и $l = 0$.
    \item Ако $V_l = V$, тогава алгоритъмът спира и търсеното покриващо дърво е $D_l$. В противен случай преминаваме към $3$.
    \item Нека $D_l(V_l, E_l)$ е дървото, построено след $l$-тата стъпка и $L_l$ са върховете от $l$-то ниво. Образуваме
    следващото ниво $L_{l+1} = \{v | v \notin V_l, \exists w \in L_l, (w, v) \in E\}$. Образуваме дървото $D_{l+1}(V_{l+1}, E_{l+1})$,
    $V_{l+1} = V_l \cup L_{l+1}$, $E_{l+1} = E_l \cup \{(w, v) | (w, v) \in E, w \in L_l, v \in L_{l+1}\}$ с присъединяване на
    всеки $v \in L_{l+1}$ към съответния $w \in L_l$. Нека $l = l + 1$ и преминаваме към $2$.
\end{enumerate}
\textbf{\underline{Алгоритъм} (обхождане в дълбочина)} \newline\newline
\underline{Дадено}: Свързан граф $G(V, E)$ \newline
\underline{Резултат}: Покриващо дърво $D(V, E')$ на $G$. \newline
\underline{Процедура}:
\begin{enumerate}
    \item Коренът $r$ на покриващото дърво ще изберем за начален връх на обхождането. Затова $L_0 = \{r\}$. Образуваме дървото
    $D_0(V_0, E_0), V_0 = \{r\}, E_0 = \varnothing$. Нека $t = r, i = 0$ и $p(t)$ е неопределен.
    \item Ако $V_l = V$, тогава алгоритъмът спира и търсеното покриващо дърво е $D_l$. В противен случай преминаваме към $3$.
    \item Нека е построено дървото $D_i(V_i, E_i)$. Търсим $v \in V_i$ такъв, че $(t, v) \in E$ и
    \begin{itemize}
        \item ако има такъв, построяваме дървото $D_{i+1}(V_{i+1}, E_{i+1}), V_{i+1} = V_i \cup \{v\}, E_{i+1} = E_i \cup \{(t, v)\}$
        Сега $p(v) = t, t = v, i = i + 1$ и преминаваме към 2.
        \item в противен случай:
        \begin{itemize}
            \item ако $t \neq r$, тогава $t = p(t)$ и преминаваме към 2.
            \item иначе - край. Търсеното покриващо дърво е $D_i$.
        \end{itemize}
    \end{itemize}
\end{enumerate}
\textbf{\underline{Дефиниция}}
\textbf{Ойлеров път} в свързания мултиграф $G(V, E)$ наричаме път, който минава еднократно през всяко ребро на мултиграфа.
Ако Ойлеровият път има съвпадащи начало и край, тогава той се нарича \textbf{Ойлеров цикъл}. \newline\newline
\textbf{\underline{Теорема}} \newline
Свързаният мултиграф $G(V, E)$ е Ойлеров $\iff$ всеки връх на $G$ е с четна степен \newline\newline
\textbf{\underline{Теорема}} \newline
Свързаният мултиграф $G(V, E)$ съдържа Ойлеров път, който не е Ойлеров цикъл $\iff$ има точно $2$ върха с нечетна степен
\begin{proof}
$\Leftarrow)$ Нека $v_i$ и $v_j$ са върховете с нечетна степен. Добавяме в мултиграфа ребро $e \notin E$ и додефинираме
$f_{G'}(e) = (v_i, v_j)$ Полученият мултиграф $G'$ е Ойлеров $\implies$ можем да построим Ойлеров цикъл. Премахваме добавеното
ребро и получаваме път от $v_i$ до $v_j$, който съдържа всички ребра на графа $G(V, E)$ точно по веднъж $\implies$ е Ойлеров път. \newline\newline
$\Rightarrow)$ Нека ребрата на $G(V, E)$ образуват Ойлеров път от $v_i$ до $v_j$. Добавяме ребро $e \notin E$ и додефинираме
$f_{G'}(e) = (v_i, v_j)$. Пътят се превръща в Ойлеров цикъл за новополучения мултиграф $G' \implies$ всички върхове са с четна
степен. Добавянето на реброто $(v_i, v_j)$ е увеличило с $1$ само степените на $v_i$ и $v_j \implies$ в $G$ всички върхове са с
четна степен, с изключение на тези два върха.
\end{proof}

\section*{Езици, автомати и изчислимост}

\subsection*{Kрайни автомати. Регулярни езици.}

\textbf{\underline{Дефиниция}}
\textbf{Краен детерминиран автомат} наричаме петорката \newline $A = <Q, \Sigma, q_0, \delta, F>$, където
$Q$ е крайно множество от състояния, $\Sigma$ е крайна входна азбука, $q_0 \in Q$ е начално състояние,
$\delta : Q \bigtimes \Sigma \rightarrow Q$ е частична функция на преходите, пресмятаща следващото състояние,
$F \subseteq Q$ са финални състояния \newline\newline
\textbf{\underline{Дефиниция}}
\textbf{Краен недетерминиран автомат} наричаме петорката \newline $A = <Q, \Sigma, q_0, \delta, F>$, където
$Q$ е крайно множество от състояния, $\Sigma$ е крайна входна азбука, $q_0 \in Q$ е начално състояние,
$\delta : Q \bigtimes \Sigma \rightarrow 2^Q$ е частична функция на преходите, пресмятаща следващото състояние,
$F \subseteq Q$ са финални състояния \newline\newline
\textbf{\underline{Теорема}} \newline
За всеки КНА $A$ съществува КДА $A'$ такъв, че $L_A = L_{A'}$
\begin{proof}
Нека $A = <Q, \Sigma, q_0, \delta, F>$. Построяваме КДА $A' = <Q', \Sigma, t_0, \delta', F'>$, където $Q' \subseteq 2^Q$. Нека множеството
$\{q_{p_1}, q_{p_2}, ..., q_{p_l}\} \in Q'$ означим с $t_{[p_1, p_2, ..., p_l]}$, където $t_0 = \{q_0\} = t_{[0]}$. Нека $F' =
\{t_{[p_1,...p_l]} | {q_{p_1}, ..., q_{p_l} \cap F \neq \varnothing}\}$ 
\end{proof}
\textbf{\underline{Дефиниция}}
Нека $\widetilde{\Sigma} = \Sigma \cup \{\varepsilon, \varnothing, *, +, (, )\}$. Дефинираме \textbf{регулярни изрази} рекурсивно:
\begin{itemize}
    \item Символите $\varnothing$, $\varepsilon$, както и всяка буква $a \in \Sigma$ са регулярни изрази
    \item Ако $r_1$ и $r_2$ са регулярни изрази, то думите $(r_1 . r_2)$, $(r_1 + r_2)$ и $(r_1)^*$ са регулярни изрази
\end{itemize}
\textbf{\underline{Дефиниция}}
Казваме, че езикът $L$ е \textbf{регулярен}, ако $L = L(r)$, за някой регулярен израз $r$ над $\Sigma$, където езикът $L(r)$ се
дефинира така:
\begin{itemize}
    \item $L(\varnothing) = \varnothing, L(a) = \{a\}$ за $a \in \Sigma$
    \item $L(r_1 . r_2) = L(r_1) . L(r_2)$, $L(r_1 + r_2) = L(r_1) \cup L(r_2)$ и $L(r_1^*) = L(r_1)^*$
\end{itemize}
\textbf{\underline{Теорема} (Клини)} \newline
Всеки език, разпознаван от краен автомат, е регулярен.
\begin{proof}
Нека $L \subseteq \Sigma^*$ е автоматен език и $\mathcal{A} = <Q, \Sigma, q_0, \delta, F>$ е краен автомат такъв, че $L(\mathcal{A}) = L$.
БОО $Q = \{0, 1, ..., n - 1\}$ и $q_0 = 0$. Нека $\forall 0 \le i, j \le n - 1$ и $\forall 0 \le k \le n$ с $R^{k}_{ij}$ да
означим множеството от всички думи над $\Sigma$, чрез които можем да се придвижим от състоянието $i$ до състоянието $j$ без да
преминаваме през състояние $s \ge k$, т.е. $$R^k_{ij} = \{w \in \Sigma^* | i \xmapsto[\mathcal{A}]{w} j, \text{ минавайки само през
състояния } s < k \}$$ Тогава $R^0_{ii} = \{\varepsilon\} \cup \{a | (i, a, i \in \delta)\}$ и $R^0_{ij} = \{a | (i, a, j) \in
\delta\}$ за $i \neq j \implies \forall 0 \le i, j \le n - 1$ езикът $R^0_{ij}$ е регулярен. От друга страна, $\forall 0 \le i, j
\le n - 1$ и $\forall 0 \le k \le n - 1$ е в сила $R^{k+1}_{ij} = R^{k}_{ij} \cup R^{k}_{ik}(R^{k}_{kk})^*R^k_{kj}$. Оттук езиците
$R^{k}_{ij}$ са регулярни $\forall 0 \le i, j \le n - 1$ и $\forall 0 \le k \le n$. От друга страна $L(\mathcal{A}) = \bigcup_{i \in F}
R^n_{0i} \implies L(\mathcal{A})$ е регулярен.
\end{proof}
\textbf{\underline{Лема} (разрастване на РЕ (uvw))} \newline
Нека $L \subseteq \Sigma^*$ е регулярен език. Тогава съществува $n \in \mathbb{N}, n \ge 1$ такова, че за всяка дума $w \in L,
|w| > n$, съществуват $x, y, z \in \Sigma^*$ такива, че $w = xyz$ и
\begin{itemize}
    \item $|xy| \le n$
    \item $|y| \ge 1$
    \item $xy^iz \in L, \forall i$
\end{itemize}
\begin{proof}
Нека $\mathcal{A} = <Q, q_0, \delta, F>$ е краен автомат над $\Sigma$ такъв, че $L(\mathcal{A}) = L$. Да положим $n = |Q|$.
Нека сега $w \in L$ с $|w| = m > n$ и нека $w = a_1...a_m$, където $a_1...a_m \in \Sigma$. Тогава съществува път $q_0, a_1, q_1,
..., a_m, q_m$ в $G_{\mathcal{A}}$ с $q_m \in F$. Тъй като $m > n$, то поне две от състоянията в горната редица съвпадат (принцип
на Дирихле). Нека $s$ е най-голямото, за което $q_i \neq q_j, \forall 0 \le i < j \le s$. Ясно е, че $s < n$ и освен това $q_k =
q_s$ за някое $k < s$. Полагаме $x = a_1...a_k, y = a_{k+1}...a_s, z = a_{s+1}...a_m$. Тогава $|xy| = s \ge n$ и $|y| = s - k
\ge 1$. При това $q_0 \xmapsto[\mathcal{A}]{x} q_k, q_k \xmapsto[\mathcal{A}]{y} q_k$ и $q_k \xmapsto[\mathcal{A}]{y} q_m$, откъдето $\forall i \ge 0, xy^iz \in L$
\end{proof}
\textbf{\underline{Примери за нерегулярни езици}}
\begin{enumerate}
    \item $\{a^nb^n \hspace{0.2cm} | \hspace{0.2cm} n \ge 0\}$
    \item $\{w \hspace{0.2cm} | \hspace{0.2cm} w = w^{Rev}\}$
    \item $\{ww \hspace{0.2cm} | \hspace{0.2cm} w \in \Sigma^*\}$
    \item $\{a^nb^nc^n \hspace{0.2cm} | \hspace{0.2cm} n \ge 0\}$
\end{enumerate}
\textbf{\underline{Теорема (Майхил-Нероуд)}} \newline
Нека $L \subseteq X*$. Релацията $R_L \subseteq X* \bigtimes X*$ има краен индекс $\iff L$ е автоматен 
\begin{proof}

\end{proof}
\subsection*{Контекстносвободни граматики и езици. Стекови автомати.}
\textbf{\underline{Дефиниция}}
\textbf{Контекстносвободна граматика} наричаме четворката $G = (\Gamma, \Sigma, R, S)$, където $\Sigma \subseteq \Gamma$ са
крайни азбуки, $S \in \Gamma \backslash \Sigma$ и $R \subseteq (\Gamma \backslash \Sigma) \bigtimes \Gamma^*$ е крайно множество.
\newline\newline
\textbf{\underline{Дефиниция}}

\textbf{\underline{Дефиниция}}
\textbf{Недетерминиран стеков автомат} наричаме седморката $A = <Q, \Sigma, Z, q_0, z_0, \delta, F>$, където
$Q$ е крайно множество от състояния, $\Sigma$ е крайна входна азбука, $Z$ е крайна стекова азбука, $q_0 \in Q$ е начално състояние,
$z_0 \in Z$ е начална стекова буква, $\delta : Q \bigtimes (X \cup \{\varepsilon\}) \bigtimes Z \rightarrow 2^{Q \bigtimes Z^*}$
е частична функция на преходите, пресмятаща следващото състояние, $F \subseteq Q$ са финални състояния \newline\newline
\textbf{\underline{Лема} (за разрастване на КСЕ)} \newline
Нека $L$ е безконтекстен език. Тогава съществува число $n > 0$ такова, че за всяка дума $w \in L, |w| > n$, съществуват думи
$x, y, z, u, v$ такива, че $w = xyzuv$ и
\begin{itemize}
    \item $|yzu| \le n, |yu| \ge 1$
    \item $xy^iuv^iw \in L, \forall i \ge 0$
\end{itemize}
\begin{proof}
Нека $G$ безконтекстна граматика в НФЧ, която генерира точно думите от $L$ с дължина поне 2. ...
\end{proof}
\textbf{\underline{Примери за неконтекстносвободни езици}}
\begin{enumerate}
    \item $\{a^nb^nc^n \hspace{0.2cm} | \hspace{0.2cm} n \ge 0\}$
    \item $\{ww^{Rev} \hspace{0.2cm} | \hspace{0.2cm} w \in \{a, b\}^*\}$
    \item $\{ww \hspace{0.2cm} | \hspace{0.2cm} w \in \{a, b\}^*\}$
    \item $\{a^nb^mc^md^n \hspace{0.2cm} | \hspace{0.2cm} n, m \ge 0\}$
\end{enumerate}

\section*{Дизайн и анализ на алгоритми}

\subsection*{Сложност на алгоритъм. Асимптотично поведение на целочислени функции 
($O$-, $\Omega$-, $\Theta$-, $o$- и $\omega$-нотация). Сложност на рекурсивни програми.}

\textbf{\underline{Дефиниция}}
\textbf{Машина на Тюринг} (с безкрайна в едната посока лента) наричаме петорката $A = <Q, \Gamma, q_0, \delta, F>$, където
$Q$ е крайно множество от състояния, $\Gamma$ е крайна азбука, $q_0 \in Q$ е начално състояние,
$\delta : Q \bigtimes \Gamma \rightarrow (Q \cup F) \bigtimes \Gamma \bigtimes \{L, R, S\}$ е функция на преходите,
$F$ са финални състояния, $F \cap Q = \varnothing$ \newline\newline
\textbf{\underline{Дефиниция}}
Нека $f_M : A^* \rightarrow A^*$ е изчислима с машината на Тюринг $M$ тотална функция. Функцията
$$t_M(n) = \underset{\alpha \in A^*, d(\alpha) = n}{max} [\text{ брой стъпки на M при работа върху $\alpha$ }]$$
наричаме \textbf{сложност по време} на $M$ в най-лошия случай, а функцията
$$\tilde{t}_M(n) = \frac{\sum_{\alpha \in A^*, d(\alpha) = n} [\text{ брой стъпки на M при работа върху $\alpha$ }]}{|A^n|}$$
наричаме средна \textbf{сложност по време} на $M$. \newline\newline
\textbf{\underline{Дефиниция}}
Нека $f_M : A^* \rightarrow A^*$ е изчислима с машината на Тюринг $M$ тотална функция. Функцията
$$s_M(n) = \underset{\alpha \in A^*, d(\alpha) = n}{max} [\text{ брой използвани клетки от M при работа върху $\alpha$ }]$$
наричаме \textbf{сложност по памет} на $M$ в най-лошия случай, а функцията
$$\tilde{s}_M(n) = \frac{\sum_{\alpha \in A^*, d(\alpha) = n} [\text{ брой използвани клетки от M при работа върху $\alpha$ }]}{|A^n|}$$
наричаме средна \textbf{сложност по памет} на $M$.

Нека $f$ е реалнозначна функция, дефинирана в $\mathbb{R}$ или $\mathbb{Z^+}$.
Функцията $f$ се нарича асимптотично отрицателна, когато $\exists n_0 \in \mathbb{N} \forall n \ge n_0 : f(n) \ge 0$
Функцията се нарича асимптотично положителна, когато $\exists n_0 \in \mathbb{N} \forall n > n_0 : f(n) > 0$

\textbf{\underline{Дефиниция}}
$\mathbf{O(g)}$ е множеството от всички функции, които растат асимптотично не по-бързо от $g$
$$O(g) = \{f \hspace{0.2cm} | \hspace{0.2cm} \exists c > 0 \hspace{0.2cm} \exists n_0 \in \mathbb{N} \hspace{0.2cm}
\forall n \ge n_0 : 0 \le f(n) \le c.g(n)\}$$ Означение: $f \in O(g), f = O(g), f \preceq g$ \newline\newline
\textbf{\underline{Дефиниция}}
$\mathbf{\Omega(g)}$ е множеството от всички функции, които растат асимптотично не по-бавно от $g$
$$\Omega(g) = \{f \hspace{0.2cm} | \hspace{0.2cm} \exists c > 0 \hspace{0.2cm} \exists n_0 \in \mathbb{N} \hspace{0.2cm}
\forall n \ge n_0 : 0 \le c.g(n) \le f(n)\}$$ Означение: $f \in \Omega(g), f = \Omega(g), f \succeq g$ \newline\newline
\textbf{\underline{Дефиниция}}
$\mathbf{o(g)}$ е множеството от всички функции, които растат асимптотично по-бавно от $g$
$$o(g) = \{f \hspace{0.2cm} | \hspace{0.2cm} \forall c > 0 \hspace{0.2cm} \exists n_0 \in \mathbb{N} \hspace{0.2cm}
\forall n \ge n_0 : 0 \le f(n) < c.g(n)\}$$ Означение: $f \in o(g), f = o(g), f \prec g$ \newline\newline
\textbf{\underline{Дефиниция}}
$\mathbf{\omega(g)}$ е множеството от всички функции, които растат асимптотично по-бързо от $g$
$$\omega(g) = \{f \hspace{0.2cm} | \hspace{0.2cm} \forall c > 0 \hspace{0.2cm} \exists n_0 \in \mathbb{N} \hspace{0.2cm}
\forall n \ge n_0 : 0 \le c.g(n) < f(n)\}$$ Означение: $f \in \omega(g), f = \omega(g), f \succ g$ \newline\newline
\textbf{\underline{Дефиниция}}
$\mathbf{\Theta(g)}$ е множеството от всички функции, които растат асимптотично еднакво с $g$
$$\Theta(g) = \{f \hspace{0.2cm} | \hspace{0.2cm} \exists c_1 > 0 \hspace{0.2cm} \exists c_2 > 0 \hspace{0.2cm} \exists n_0
\in \mathbb{N} \hspace{0.2cm} \forall n \ge n_0 : 0 \le c_1.g(n) \le f(n) \le c_2.g(n)\}$$ Означение: $f \in \Theta(g),
f = \Theta(g), f \asymp g$ \newline\newline

\textbf{\underline{Свойства}}
\begin{enumerate}
    \item Ако $f \sigma g$ и $g \sigma h$, то $f \sigma h, \hspace{0.2cm} \sigma \in \{\prec, \succ, \preceq, \succeq, \asymp\}$ -
    транзитивност
    \item $f \sigma f, \hspace{0.2cm} \forall \sigma \in \{\preceq, \succeq, \asymp\}$ ($\Theta$, $O$ и $\Omega$ са рефлексивни)
    \item $f \succeq g$ и $g \succeq f \iff f \asymp g$ ($O$ и $\Omega$ са антисиметрични)
    \item $f \asymp g \implies g \asymp f$ ($\Theta$ е симетрична)
    \item $f \preceq g \iff g \succeq f, f \prec g \iff g \succ f$
\end{enumerate}
\textbf{\underline{Теореми} (гранични)} \newline
$$max\{f, g\} \asymp f + g$$
$\lim_{n \rightarrow \infty}\frac{f(n)}{g(n)} = 0 \iff f = o(g) \hspace{1cm} \lim_{n \rightarrow \infty}\frac{f(n)}{g(n)} =
\infty \iff f = \omega(g)$
$$\lim_{n \rightarrow \infty}\frac{f(n)}{g(n)} = c > 0 \implies f = \Theta(g)$$ \newline\newline
\textbf{\underline{Теорема} (Мастър)} \newline
Нека $a \ge 1, b > 1$ и $f(n)$ и $T(n)$ са асимптотично положителни функции и $T(n) = a.T(\frac{n}{b}) + f(n)$. Нека
$k = \log_ab$. Тогава:
\begin{enumerate}
    \item Ако $f(n) = O(n^{k - \varepsilon})$ за някое $\varepsilon > 0$, то $T(n) = O(n^k)$
    \item Ако $f(n) = O(n^k)$, то $T(n) = \Theta(n^klogn)$
    \item Ако $f(n) = \Omega(n^{k + \varepsilon})$, за някое $\varepsilon > 0$ и $\exists c \in (0, 1) \hspace{0.2cm}
    \exists n_0 \in \mathbb{N} \hspace{0.2cm} \forall n \ge n_0 : a.f(\frac{n}{b}) \le c.f(n)$, то $T(n) = \Theta(f(n))$
\end{enumerate}

\subsection*{Алгоритми в графи с тегла на ребрата. Оценки за сложност.}
\textbf{\underline{Дефиниция}}
Нека $G(V, E)$ е свързан граф и $c : E \rightarrow R$ е функция с реални стойности, дефинирана по ребрата на $G$. Стойността
$c(e), e \in E$ наричаме \textbf{тегло} (цена) на реброто $e$. Нека $D(V, E')$ е покриващо дърво на $G(V, E)$. \textbf{Тегло}
(цена) на дървото наричаме сумата $c(D) = \Sigma_{e \in E'}c(e)$. Покриващото дърво $D_0(V, E_0)$ на $G(V, E)$ наричаме
\textbf{минимално}, ако $c(D_0) \le c(D)$ за всяко друго покриващо дърво $D$ на $G$. \newline\newline
\textbf{\underline{Теорема} (МПД-свойство)} \newline
Нека $G(V, E)$ е свързан граф с теглова функция на ребрата $c : E \rightarrow R$ и $\varnothing \neq U \subset V$. Нека $e = (v_i,
v_j) \in E$ е такова, че $v_i \in U, v_j \in V \backslash U$ и $(v_i, v_j)$ има минимално тегло измежду всички такива ребра.
Тогава съществува МПД $D(V, E')$ на $G$ такова, че $e \in E'$ \newline\newline
\textbf{\underline{Алгоритъм} (Прим)} \newline\newline
\underline{Дадено}: Свързан граф $G(V, E)$ и функция $c : E \rightarrow R$, задаваща тегла на ребрата му. \newline
\underline{Резултат}: Минимално покриващо дърво $D(V', E)$ с корен - зададен връх $r \in V$ на $G$. \newline
\underline{Процедура}:
\begin{enumerate}
    \item Построяваме дървото $D_0(V_0, E_0), V_0 = \{r\}, E_0 = \varnothing, k = 0$.
    \item Нека сме построили $D_k(V_k, E_k)$. Търсим реброто $e = (v_i, v_j), v_i \in V_k, v_j \in V \backslash V_k$
    с минимално тегло и построяваме $D_{k+1}(V_{k+1}, E_{k+1}), V_{k+1} = V_k \cup \{v_j\}, E_{k+1} =
    E_k \cup \{\varepsilon\}, k = k + 1$
    \item Ако $V_k = V$ - край, полученото дърво $D(V, E'), E' = E_k$ е оптималното. Иначе минаваме към 2.
\end{enumerate}
\underline{Оценка на сложността}: $T = O(V + E), S = O(V + E)$ \newline\newline
\textbf{\underline{Алгоритъм} (Крускал)} \newline\newline
\underline{Дадено}: Свързан граф $G(V, E)$ и функция $c : E \rightarrow R$, задаваща тегла на ребрата му. \newline
\underline{Резултат}: Минимално покриващо дърво $D(V', E)$ на $G$. \newline
\underline{Процедура}:
\begin{enumerate}
    \item Сортираме ребрата на $G$ в нарастващ ред на цената и нека този ред е $e_1, ..., e_m$
    \item От всеки връх $v$ на графа образуваме тривиално дърво $D_v(\{v\}, \varnothing)$
    \item За всяко ребро $e_i = (v_{i_1}, v_{i_2}), i \in I_m$ (по реда определен от сортирането), правим следното:
    ако $v_{i_1}, v_{i_2}$ са в различни дървета, съответно $D'(V', E')$ и $D''(V'', E'')$, съединяваме двете в
    дървото $D(V' \cup V'', E' \cup E'' \cup \{(v_{i_1}, v_{i_2})\})$
\end{enumerate}
\underline{Оценка на сложността}: $T = O(V + E), S = O(V + E)$ \newline\newline
\textbf{\underline{Алгоритъм} (Дейкстра)} \newline\newline
\underline{Дадено}: Свързан граф $G(V, E)$ и функция $c : E \rightarrow R^+$ и начален връх $v_0 \in V$. \newline
\underline{Резултат}: Дърво на минималните пътища от $v_0$ до всички останали върхове в $G$. \newline
\underline{Процедура}:
\begin{enumerate}
    \item Разширяваме $c : E \rightarrow R^+$ до $c* : V \bigtimes V \rightarrow R^+$.
    \item Нека $dist[0] = 0, part[0] = -1$ и $U = \{0\}$, а $dist[i] = c*(0, i)$ и $part[i] = 0, i \in I_n$
    \item Повтаряме $n - 1$ пъти стъпките:
    \begin{itemize}
        \item Избираме връх $j \notin U$, за който $dist[j]$ е минимално и $U = U \cup \{j\}$
        \item За всеки $k \notin U$ пресмятаме $dist[k] = min(dist[k], dist[j] + c*(j, k))$.
        Ако $min$ е $dist[j] + c*(j, k)$, тогава $part[k] = j$
    \end{itemize}
\end{enumerate}
\underline{Оценка на сложността}: $T = O(V + E), S = O(V + E)$ \newline\newline
\textbf{\underline{Алгоритъм} (Флойд)} \newline\newline


\subsection*{Динамично програмиране. Оценки за сложност.}

Динамичното програмиране е метод за решаване на задачи чрез комбиниране на решенията на подзадачи. Алгоритъмът решава всяка
подзадача веднъж и запазва резултата в таблица, избягвайки нуждата от преизчисляване на отговора всеки път, когато решава някоя
подзадача. \newline\newline
\textbf{\underline{Задачи с линейна таблица на подзадачите} (най-дълга растяща подредица)} \newline\newline
\textbf{\underline{Задачи с триъгълна таблица на подзадачите} (оптимално разбиване на подредица)} \newline\newline
\textbf{\underline{Задачи с правоъгълна таблица на подзадачите} (най-дълга обща подредица на две редици, задача за раницата)} \newline\newline

\end{document}